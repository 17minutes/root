\documentclass[11pt,twoside]{myArticle}       % CUSTOMIZED (!) article

%
% use PDFLatex
%
\pdfoutput=1

\usepackage{amsmath,amssymb,amsfonts} % Typical maths resource packages
\usepackage{graphicx}
\usepackage{epsfig}
\usepackage{color}                    % For creating coloured text and background
\usepackage{hyperref}                 % For creating hyperlinks in cross references
\usepackage{myFancyhdr}               % CUSTOMIZED (!) fancy headers and footer
\usepackage{makeidx}                  % Package for indexing
\usepackage{helvet}                   % For sans-serif mode
\usepackage{setspace}                 % Allows to customize line distances
\usepackage{fancybox}                 % For code example and option boxes
\usepackage{float}                    % Floating objects
\usepackage{xspace}                   % ...
\usepackage{wasysym}                  % permille sign
\usepackage[numbers,sort&compress]{natbib}
\usepackage{hypernat}
\usepackage{subfigure}
\usepackage{rotating}                 % for sideways tables/figures
% \usepackage{times}
%
% define TMVA and document versions ==========================================
\newcommand\TMVAVersion{4.2.0\xspace}
% ============================================================================
%
% setup colors (requires usepackage{color}
%
\definecolor{darkblue1}{rgb}{0,0,.7}
\definecolor{darkblue}{rgb}{0,0,.3}
\definecolor{darkred}{rgb}{0.5,0,0}
\pagecolor{white} % Background color
\color{black}     % Text color
%
% setup hyper links
%
\hypersetup{breaklinks=true, 
            colorlinks=true, 
            linkcolor=darkblue1, 
            menucolor=darkblue1, 
            urlcolor=darkblue1,
            citecolor=darkblue1,
            pdftitle={TMVA - Users Guide},
            pdfauthor={TMVA},
            pdfsubject={TMVA - Users Guide},
            pdfkeywords={},
            pdfproducer={TMVA}
}
%
% setup page margins and style
%
\parsep 20ex
\parskip 0.04cm

\topmargin -0.5cm
\oddsidemargin -0.1cm
\evensidemargin -0.1cm
\textwidth 16.6cm
\textheight 21.5cm
\parindent 0cm % no indentation throughout document
%
% setup natbib
%
\bibstyle{plain}
%
% ------------ include custom definitions
%
\input TMVAdef
%
% create index
%
\makeindex
%
% Forces the page to use the fancy template
%
\pagestyle{fancy}  
%
% line gap between paragraphs
%
\parindent 0cm % no indentation throughout document
\parskip 1.6ex

% ============================================================================
\begin{document}
% ============================================================================

%
% ------------ title
%
\input Title
%
% ------------ contents, figures, tables
%
\pagenumbering{roman}

\setcounter{tocdepth}{3}
\addtolength{\parskip}{-0.40\baselineskip}
{\footnotesize\sl
\twocolumn
\tableofcontents
\onecolumn
}
\addtolength{\parskip}{0.40\baselineskip} 
% \listoffigures
% \listoftables

\newpage

\pagenumbering{arabic}
%
% ------------ introduction
%
\input Introduction
%
% ------------ technicalities
%
\input UsingTMVAQuickStart
\input UsingTMVA
\clearpage
%\vfill\pagebreak
\input DataPreprocessing
\input CommonTools
%\vfill\pagebreak
\clearpage
%
% ------------ MVA methods
%
\input MethodsIntro
%
\input Cuts
\input Likelihood
\input PDERS
\input PDEFoam
\input KNN
\input HMatrix
\input Fisher
\input LD
\input FDA
\input MLPs
\input SVM
\input BDTs
\input RuleFit
\input Combining
%
% ------------ conclusions
% 
\input Conclusions
%
% ------------ references
% 
\newpage
\input Appendix
%
% ------------ references
% 
\newpage
\input Bibliography
%
% ------------ print and include index
%
\addcontentsline{toc}{section}{Index}
\printindex

% ============================================================================
\end{document}
% ============================================================================

